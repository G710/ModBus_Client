
\chapter{Settings}

The settings tab contains operating parameters,
beware that the settings are tied to the profile,
changing the profile loads the respective settings.

\begin{figure}[H]
    \centering
    \includegraphics[width=0.85\textwidth]{../Img/Modbus_Client_Settings_00.PNG}
    \caption{Settings tab}
\end{figure}

\begin{itemize}
    \item \textbf{Automatically correct offset modbus addresses:}
    If checked when entering an address in the original protocol format e.g.
    holding register 40002 is automatically sent in the modbus request as address
    00001 thus eliminating the offset provided for holding registers. Otherwise it is 
    requested
    address 40002. Same for input registers (30001), discrete inputs (10001)
    or coils (00001).
    \item \textbf{Color cells with value > 0 / Alternating color:}
    If checked, only table rows with a value > 0 are colored, otherwise rows
    are colored alternately.
    \item \textbf{Hide user offset in tables:}
    If checked, the general offset is not displayed in the tables but is counted in the commands
    sent via ModBus. (User offset means the offset entered in the box "Offset" present
    in each tab).
    \item \textbf{Close console on startup:}
    Automatically closes the console when you start the application.
    \item \textbf{Dark mode:}
    Enable dark mode (black backgrounds, white text).
\end{itemize}

In addition to the ticks there are some parameters such as:

\begin{itemize}
    \item \textbf{Read timeout:}
    Timeout in ms for response to a command.
    \item \textbf{Loop polling interval:}
    Interval in ms between two reads for loop read commands.
\end{itemize}

For non-secure TCP connections only, it is possible to choose between two modes
of socket management:

\begin{itemize}
    \item \textbf{TCP connection mode:}
    This parameter allows you to change the TCP socket handling only for 
    non-secure mode. At the TCP socket level normally the connection is opened,
    kept open and closed at the end of its use.
    On unstable connections, however, it is more convenient to open and 
    close the socket on each request.
    The correct mode of the protocol would be the first one,
    but it often happens 
    on unstable connections (e.g., slaves connected via modem) that a connection is initiated,
    after a few reads it falls out, the socket fails and you have to reconnect again.
    With the socket going into timeout as it is no longer open on the slave 
    when the connection returns. With the second
    mode this does not happen because the socket is opened for each read 
    and closed immediately after so on a practical level this mode can come in very handy.
\end{itemize}

Preview import settings:
\\\\
The following settings allow you to set the default values
of the preview import window.

\begin{itemize}
    \item \textbf{WRITE MULTIPLE REGISTERS (FC15 / FC16):} merge consecutive records into a single command where possible.
    \item \textbf{CLOSE WINDOW AFTER WRITE:} closes the window automatically when the writing is finished.
    \item \textbf{ABORT WRITE ON ERROR:} stops the entries in case of an error (otherwise ignores the error and continues with the next entries).
    \item \textbf{NR. OF REGISTERS MULTIPLE WRITE:} maximum number of registers sent in a single command FC15 / FC16.
\end{itemize}