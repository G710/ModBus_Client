
\chapter{Tasti di scelta rapida}

\begin{itemize}
\item \textbf{Ctrl + Num 1 - 9}: Seleziona la tab con l'indice selezionato dal numero premuto
\item \textbf{Ctrl + Q/Ctrl + W}: Chiude la finestra
\item \textbf{Ctrl + E}: Legge il range di registri della tab selezionata
\item \textbf{Ctrl + R}: Legge i registri della tab selezionata (inputs/coils/input registers/holding registers)
\item \textbf{Ctrl + T}: Apre la finestra di modifica template per il profilo attualmente selezionato
\item \textbf{Ctrl + Y}: Apre finestra di import csv/json
\item \textbf{Ctrl + U}: Apre finestra di export csv/json
\item \textbf{Ctrl + I}: Apre finestra info versione software
\item \textbf{Ctrl + O}: Apre il menu per caricare un profilo
\item \textbf{Ctrl + P}: Avvia/Ferma il polling per la tab selezionata (pulsante loop)
\item \textbf{Ctrl + S}: Salva eventuali modifiche al profilo attualmente selezionato
\item \textbf{Ctrl + Shift + S}: Apre il menu per salvare il profilo corrente come nuovo profilo
\item \textbf{Ctrl + D}: Apre la finestra di gestione del database
\item \textbf{Ctrl + F}: Attiva/disattiva modalità schermo intero tabelle
\item \textbf{Ctrl + G}: Legge il gruppo selezionato nel menu a tendina per la tab corrente
\item \textbf{Ctrl + H}: Legge tutte le risorse inserite a template per la tab corrente
\item \textbf{Ctrl + J}: Apre la finestra delle statistiche
\item \textbf{Ctrl + K}: Avvia/Ferma il polling per la tab selezionata sul range indicato (pulsante loop)
\item \textbf{Ctrl + L}: Apre la finestra di log
\item \textbf{Ctrl + Shift + C}: Apre/chiude console di debug client
\item \textbf{Ctrl + B}: Connetti/Disconnetti
\item \textbf{Ctrl + M}: Passa da TCP a RTU e viceversa
\item \textbf{Del}: Cancella contenuto tabelle della tab corrente
\end{itemize}