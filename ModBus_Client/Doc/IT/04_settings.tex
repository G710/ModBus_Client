
\chapter{Impostazioni}

La tab impostazioni contiene i parametri di funzionamento,
attenzione che le impostazioni sono legate al profilo,
cambiando profilo vengono caricate le rispettive impostazioni.

\begin{figure}[H]
    \centering
    \includegraphics[width=0.85\textwidth]{../Img/Modbus_Client_Settings_00.PNG}
    \caption{Tab impostazioni}
\end{figure}

\begin{itemize}
    \item \textbf{CORREGGI AUTOMATICAMENTE OFFSET INDIRIZZI MODBUS TEXTBOX:}
    Se spuntata quando si inserisce un indirizzo nel formato originale del protocollo ad esempio
    holding register 40002 viene automaticamente inviato nellla richiesta modbus come indirizzo
    00001 eliminando quindi l'offset previsto per gli holding registers. In caso contrario viene 
    richiesto
    l'indirizzo 40002. Stessa cosa per gli input registers (30001), discrete inputs (10001)
    o coils (1).
    \item \textbf{COLORE CELLE > 0 / COLORE CELLE ALTERNATE:}
    Se spuntata vengono colorate solo le righe delle tabelle con un valore > 0, altrimenti le righe
    vengono colorate in maniera alternata.
    \item \textbf{NASCONDI OFFSET UTENTE NELLE TABELLE:}
    Se spuntata l'offset generale non viene visualizzato nelle tabelle ma viene contato nei comandi
    inviati via ModBus. (Per offset utente si intende l'offset inserito nel box presente
    in ogni tab).
    \item \textbf{CHIUDI CONSOLE ALL'AVVIO:}
    Chiude automaticamente la console quando si avvia l'applicazione.
    \item \textbf{DARK MODE:}
    Abilita la modalità scura (sfondi neri, testo bianco).
\end{itemize}

Oltre alle spunte sono presenti alcuni parametri quali:

\begin{itemize}
    \item \textbf{READ TIMEOUT:}
    Timeout in ms di risposta a un comando.
    \item \textbf{LOOP POLLING INTERVAL:}
    Intervallo in ms tra una lettura e la successiva per i comandi di loop read.
\end{itemize}

Solo per connessioni TCP non secure è possibile scegliere tra due modalità
di gestione delle socket:

\begin{itemize}
    \item \textbf{TCP CONNECTION MODE:}
    Questo parametro permette di cambiare la gestione della socket TCP solo per la 
    modalità non secure. A livello di socket TCP normalmente la connessione viene aperta,
    mantenuta aperta e chiusa al termine del suo utilizzo.
    Su connessioni instabili però risulta più comodo aprire e 
    chiudere la socket ad ogni richiesta.
    La modalità corretta del protocollo sarebbe la prima,
    ma succede spesso 
    su connessioni instabili (ad esempio slave connessi via modem) che si inizi la connessione,
    dopo qualche lettura cade la connessione, va in errore la socket e bisogna connettersi nuovamente
    con la socket che va in timeout non essendo più aperta sullo slave 
    quando ritorna la connessione. Con la seconda
    modalità questo non accade perche la socket viene riaperta per ogni lettura 
    e chiusa subito dopo per cui a livello pratico questa modalità può tornare molto utile.
\end{itemize}

Preview import settings:
\\\\
Le impostazioni seguenti permettono di impostare i valori di default
della finestra di import dei registri.

\begin{itemize}
    \item \textbf{WRITE MULTIPLE REGISTERS (FC15 / FC16):} accorpa registri consecutivi in un unico comando ove possibile.
    \item \textbf{CLOSE WINDOW AFTER WRITE:} chiude la finestra automaticamente al termine delle scritture.
    \item \textbf{ABORT WRITE ON ERROR:} interrompe le scritture in caso di errore (altrimenti ignora l'errore e prosegue con le successive).
    \item \textbf{NR. OF REGISTERS MULTIPLE WRITE:} numero massimo di registri inviati in un singolo comando FC15 / FC16.
\end{itemize}